\documentclass[a4paper,12pt]{article}
\author{Eduard}
\title{Proyecto de Programación I.Facultad de Matemática y Computación - Universidad de La Habana.}

\begin{document}
\maketitle
Moogle! es una aplicaci\'on web cuyo prop\'osito es buscar inteligentemente un
texto en un conjunto de documentos.Est\'a desarrollada con tecnolog\'ia .NET Core 6.0,
específicamente usando Blazor como "framework" web para la interfaz gr\'afica, y en el lenguaje C Sharp.\\
 La aplicación está dividida en dos componentes fundamentales:

- "MoogleServer" es un servidor web que renderiza la interfaz gr\'afica y sirve los resultados.\\
- "MoogleEngine" es una biblioteca de clases donde está implementada toda la l\'ogica del programa

La b\'usqueda se realiza lo m\'as inteligente posible, por ese motivo no nos limitamos a 
los documentos donde aparece exactamente la frase introducida por el usuario, sino que tambi\'en se realiza una b\'usqueda por parecido entre las palabras.\\
 

El proyecto por defecto tiene tres clases; una clase Moogle donde voy a implementar lo necesario para
que el programa busque y dos clases que me ayudan a organizar esos resultados de la búsqueda .La
primera es searchitem para caracterizar a un resultado de la búsqueda la misma tiene tres
propiedades ,el title snippet score esta última es que tanto se parece a lo que estoy buscando y es lo que 
utilizo para ordenar los resultados en relación a la concordancia que tienen o la similitud que tienen con
la query , el snippet es un fragmento que voy a devolver junto con ese resultado mostrando que ese
resultado que devuelve tiene realmente que ver con mi query, el title es el título del libro que estoy
devolviendo.
Para este proyecto implemente las siguentes clases comenzando por la clase Book  para guardar todo lo
referente a la información necesaria para la búsqueda de un libro o sea todo lo que se necesita sacar de
un libro, clase Library será mi biblioteca para guardar la información de varios libros para tener más
información creando así un conjunto de documentos más organizados, clase toolsBox que contiene
métodos útiles que se usarán más adelante para poder realizar la búsqueda estos códigos son
generalmente recurrente para usarlos varias veces llamándolos desde cualquier otro lugar desde
cualquier clase, SearchEngine este sería el motor de busqueda para construir la biblioteca para poder
tener toda la información almacenada aquí se realizan todos los procesos para poder realizar la
búsqueda es donde también esta implementado el modelo vectorial utilizado para poder dar los
resultados de acuerdo a la búsqueda y en el método search se construyen los resultados de búsqueda,
método que llama a todos los demás métodos ya sea de las demás clases y de la propia clase en sí para
poder construir el resultado que queremos mandar, devolviendo un serchresult que en la clase Moogle
simplemente tenemos que generar una instancia de la clase SearchEngine iniciarla y preguntarle a esa
instancia por los resultados de la búsqueda, a través del método Search.\\

El modelo vectorial se basa en lo siguiente:\\
Toma la consulta, que consiste en un conjunto de palabras y la convierte en el vector $\vec{q} = ( w_1 , w_2 , \dots , w_n )$ donde cada componente se refiere al peso de una palabra de la consulta
Luego, usando la mismas palabras de la consulta se construyen los vectores $\vec{d_j} = ( w_{1j} , w_{2j} , \dots , w_{nj})$ con $j = 1;2; \dots ; D$ siendo $D$ la cantidad de documentos sobre los que se har\'a la b\'usqueda.\\
Los valores de las componentes de cada vector se calculan seg\'un las f\'ormulas siguientes:\\
\begin{eqnarray}
    w_i = (a + (1- a)*\frac{TF_i}{max TF})*\log_{10}(\frac{D}{D_i})\\
    w_{ji} = \frac{TF_{ij}}{max TF_j} * \log_{10}(\frac{D}{D_i})
\end{eqnarray}
Donde $TF_i$ es la cantidad de veces que se repite la palabra $i$ en la consulta, $maxTF$ es la cantidad de veces que se repite la palabra que m\'as se repite en la consulta, $a$ es 
un valor de amoriguamiento que se en cuentra en el intervalo $(0;1)$; se suele escoger un valor entre $0.5$; y $D_i$ es la cantidad de documentos que contienen a la palabra $i$. $TF_{ij}$
es la cantidad de veces que se repite la palabra $i$ en el documento $j$ y $maxTF_j$ es la cantidad de veces que se repite la palabra que m\'as se repite en el documento $j$. Cada uno de 
estos valores pueden ser obtenidos de la clase Library ya que son datos que se computan directamente de los documentos, y estos datos se computan a la hora de cargar todos los documentos
para la aplicaci\'on.\\
\\
De la asignatura de \'Algebra conocemos que:
\begin{eqnarray}
    \vec{q}*\vec{d_j} = \sum_{i = 1}^{n} w_i*w_{ji}\\
    \vec{q}*\vec{d_j} = |\vec{q}|*|\vec{d_j}|*\cos\angle(\vec{q},\vec{d_j})\\
    |\vec{q}| = \sqrt{\sum_{i = 1}^n w_i^2}\\
    |\vec{d_j}| = \sqrt{\sum_{i = 1}^n w_{ji}^2}
\end{eqnarray}
Sustituyendo $(3),(5),(6)$ en $(4)$ se obtiene:
\begin{eqnarray}
    \sum_{i = 1}^n w_i*w_{ji} = \cos\angle(\vec{q},\vec{d_j}) *\sqrt{(\sum_{i = 1}^n w_i^2 )* (\sum_{i = 1}^n w_{ji}^2)} 
\end{eqnarray}
y despejando $\cos\angle(\vec{q},\vec{d_j})$ en $(7)$ obtenemos:
\begin{eqnarray}
    \cos \angle(\vec{q},\vec{d_j}) = \frac{\sum_{i = 1}^n w_i*w_{ji}}{\sqrt{(\sum_{i = 1}^n w_i^2 )* (\sum_{i = 1}^n w_{ji}^2)}}
\end{eqnarray}
Donde $\cos \angle(\vec{q},\vec{d_j})$ es la similitud entre la consulta $\vec{q}$ y el documento $\vec{d_j}$

Para una mayor eficiencia de la aplicaci\'on, cada dato es almacenado en una matriz de la siguiente forma; una matriz con el tf de cada palabra de la consulta por cada documento,

\begin{center}
    \left( \begin{tabular}{cols}
        $\frac{TF_{11}}{maxTF_1}$ & $\frac{TF_{12}}{maxTF_1}$ & \dots & $\frac{TF_{1n}}{maxTF_1}$\\
        $\frac{TF_{21}}{maxTF_2}$ & $\frac{TF_{22}}{maxTF_2}$ & \dots & $\frac{TF_{2n}}{maxTF_2}$\\
        \dots & \dots & \dots & \dots\\
        $\frac{TF_{m1}}{maxTF_m}$ & $\frac{TF_{m2}}{maxTF_m}$ & \dots & $\frac{TF_{nm}}{maxTF_m}$\\
    \end{tabular}\right)
\end{center}
\\
\\
y un vector con el idf de cada palabra de la consulta en el conjunto de documentos $\vec{p} = (p_1 , p_2 , \dots , p_n)$ donde se cumple que $p_i = \log_{10}(\frac{D}{D_i}$); 
de esta forma, aplicar los operadores se reduce a alterar los valores de las componentes del vector $\vec{p}$ o las componentes de la matriz presentada.\\
Luego, cada componente del vector $\vec{p}$ se multiplica por la columna correspondiente, obteniendo as\'i la matriz\\
$W$ = 
\begin{left}
    \left( \begin{tabular}{cols}
        $w_{11}$ & $w_{12}$ & \dots & $w_{1n}$\\
        $w_{21}$ & $w_{22}$ & \dots & $w_{2n}$\\
        \dots & \dots & \dots & \dots\\
        $w_{m1}$ & $w_{m2}$ & \dots & $w_{mn}$\\
    \end{tabular}\right)
\end{left}
\\
\\
Luego, haciendo\\
\\
 $\vec{q}$ =
\begin{left}
    \left( \begin{tabular}{cols}
        $w_1$\\
        $w_2$\\
        \dots\\
        $w_{n}$\\
    \end{tabular}\right)
\end{left} 
\\
\\
si efectuamos el producto $W*\vec{q}$ =
\begin{left}
    \left( \begin{tabular}{cols}
        $w_{11}$ & $w_{12}$ & \dots & $w_{1n}$\\
        $w_{21}$ & $w_{22}$ & \dots & $w_{2n}$\\
        \dots & \dots & \dots & \dots\\
        $w_{m1}$ & $w_{m2}$ & \dots & $w_{mn}$\\
    \end{tabular}\right)
\end{left} * 
\begin{left}
    \left( \begin{tabular}{cols}
        $w_1$\\
        $w_2$\\
        \dots\\
        $w_{n}$\\
    \end{tabular}\right)
\end{left} 
\\
\\se obtiene \theta  =
\begin{left}
    \left( \begin{tabular}{cols}
        $\sum_{i = 1}^n w_i * w_{1i}$\\
        $\sum_{i = 1}^n w_i * w_{2i}$\\
        \dots\\
        $\sum_{i = 1}^n w_i * w_{mi}$\\
    \end{tabular}\right)
\end{left}
= 
\begin{left}
    \left( \begin{tabular}{cols}
        \theta_1\\
        \theta_2\\
        \dots\\
        \theta_m\\
    \end{tabular}\right)
\end{left}
\\
\\Sea \\
$p_j = \sqrt{(\sum_{i = 1}^n w_i^2 )* (\sum_{i = 1}^n w_{ji}^2)}$
\\
\\
si hacemos \theta^{'} = 
\begin{left}
    \left( \begin{tabular}{cols}
        $\frac{\theta_1}{p_1}$\\
        $\frac{\theta_2}{p_2}$\\
        \dots\\
        $\frac{\theta_m}{p_m}$\\
    \end{tabular}\right)
\end{left}
\\
\\
entonces $ \theta^{'}$ ser\'a el vector con las similitudes de cada documento con la consulta despu\'es de haber aplicado los operadores de b\'usqueda a toda la consulta.
\end{document}
